\section{Output Probability Computations}
The output probability expression for a Boolean function is a real-valued
algebraic equation that specifies the probability that the function 
is valued at logic-`1' given the probabilities that each of the dependent variables
are valued at logic-`1'.  
Therefore, the probability
space consists of $2^n$ experiments where $n$ is the number of dependent
Boolean variables.  
If it is assumed that the function is fully specified,
and that each input is equally likely to be 0 or 1, all probabilities for
function variables may be set to $\frac{1}{2}$, and the resulting circuit
output probability will have a value in the interval, $[0,1]$.
This resulting probability is simply the percentage of minterms that
cause the function to evaluate to logic-`1'.

The output probabilities were first proposed by Parker and McCluskey
in the work \cite{PM75} where they were used to evaluate the effectiveness of
random testing for combinational logic circuits.  Recently, interest has been
renewed in these quantities 
since they can be used to form estimates of switching activity factors
under
mild assumptions.  Switching activity factors are very
useful in the prediction of overall power dissipation for CMOS 
circuitry and are thus a quantity to be minimized for low 
power design \cite{GDKW92} \cite{MP96} \cite{TPD93}.

In the original paper \cite{PM75} where output probabilities were defined, 
two methods were presented for the computation
of the output probability expressions for a logic circuit.  The first method
consisted of using algebraic expressions and the second relied upon a schematic
diagram.  Later, a new method was developed for the computation of these quantities
using the decision diagram data structure \cite{TN94e}.  In particular, the 
ordered binary decision diagram (OBDD) proposed by Bryant \cite{RB86} was used
along with algorithms for OBDD manipulation to compute individual output
probability values.  Most recently, the method in \cite{TN94e} has been improved
through the use of shared binary decision diagrams (SBDDs) and edge negations
as described in \cite{DM95}.

Any of the methods mentioned above may be used to determine
the percentage of minterms that cause a function, $f$, to evaluate to logic-`1'.
This quantity is denoted here as p(f).  As an example, consider the function
defined by the Boolean expression in Equation \ref{eq:ex1}.

\begin{equation}
f(x) = x_1 x_3 \overline{x}_6 + x_1 \overline{x}_3 x_4 \overline{x}_6
+ x_1 \overline{x}_3 \overline{x}_4 \overline{x}_5
+ \overline{x}_1 x_2 x_4 \overline{x}_6
+ \overline{x}_1 x_2 \overline{x}_4 \overline{x}_5
+ x_1 \overline{x}_2 \overline{x}_5    \label{eq:ex1}
\end{equation}

Application of the techniques described above yield the result, p(f) = \frac{5}{8}$.
Thus, 62.5 \% of the possible $2^6$ minterms will cause $f$ to evaluate to a logic-`1' value.

