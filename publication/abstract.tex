\begin{abstract}
A Boolean function may be uniquely represented by a
complete spectral vector or a truth table.  The former
provides global information regarding the function behavior
over all inputs while the latter provides local 
information specifying the functions behavior for a
specific set of variable values.  
This paper examines the relationship 
between the spectral and Boolean domains and shows that
various forms of output probabilities may be considered as
fundamental quantities.  
This result is important since it
demonstrates the existence of a unifying set of values
that may be used to determine function characteristics in either
the Boolean or spectral domains.
\end{abstract}
