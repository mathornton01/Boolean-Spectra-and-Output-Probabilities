\section{Conclusion}
This paper has examined the relationship between the output probabilities of
Boolean functions and their properties in both the Boolean and spectral domains.
The output probabilities have been shown to be directly related to quantities of
interest such as cube covering, Boolean derivatives, consensus, and smoothing
operations as well as various spectral coefficients including the Walsh family,
Reed-Muller family, and Haar varieties.  Thus, it is concluded that output probability
values provide a very general measure of various Boolean function properties, and, 
that an efficient means for the calculation of the probability values implies that
certain Boolean and spectral measures are also easily computed.

In the future, we plan to investigate problems that require intensive computation
in either the Boolean or spectral domains to determine if the use of output
probabilities can lead to some simplification.
