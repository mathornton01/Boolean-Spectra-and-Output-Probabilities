\section{Computation of the Spectra Using Output Probabilities}
This section will describe how output probabilities can be used to compute
various spectral coefficients directly.  The key idea here is to view
the computation in terms of a vector-matrix product using an appropriate
transformation matrix.  Although such products are not computed directly,
with this viewpoint each transformation matrix row vector can be considered
as an output vector of some other function called a `constituent function'.

By using Boolean relations between the function to be transformed, $f$, and
the various constituent functions, $f_c$, output probabilities may be computed and
used to calculate various spectral coefficients through simple algebraic relations.  
The advantage of this approach is that
the spectral coefficients are computed individually allowing complete storage of
$2^n$ spectral coefficients to be avoided.  Furthermore, if efficient methods are
used to compute the probability values a savings in computation time also
results.

\subsection{Walsh Spectrum}
The Walsh spectral values form several different spectra.  The chief difference
in the various spectra depend only upon the order that coefficients appear
\cite{HMM85}.  For example, this set of coefficients may be ordered naturally,
in sequency order, in dyadic order, or, in revised sequency order.  
These orders yield the Hadamard-Walsh,
the Walsh, the Paley-Walsh, and, the Rademacher-Walsh spectra respectively \cite{PB85} \cite{HMM85}.
In the past, the Hadamard-Walsh has seen much use due to the fact that efficient
decompositions of the transform matrix may be obtained allowing for `fast
transform' methods \cite{CT65} \cite{JS69} to be applied.

Since the relationships described here apply only to a single Walsh coefficient,
the particular coefficient orderings are not important.  Each coefficient
is described by the particular $f_c$ that corresponds to a transformation
matrix row vector regardless of the actual location of the row vector in the matrix.
The Walsh coefficient is denoted by $W_f(f_c)$ where $f$ denotes the coefficient
is with respect to function $f$ and $f_c$ denotes it is the coefficient resulting
from the row vector corresponding to constituent function $f_c$.

It is generally the case that the Walsh spectral values are computed by replacing
all occurrences of logic-1 with the integer, -1, and all occurrences of logic-0
with the integer, +1.  The actual calculation of the coefficient is then
carried out by using integer arithmetic.  In terms of computing an inner-product
of a transformation row vector and an output vector of a function to be transformed,
it is easy to see that each scalar product to be accumulated is either +1 or -1 in value.
Furthermore, a scalar product of -1 will only occur when the functions $f$ and $f_c$ have
different output values for the same set of input variable assignments.

If we consider the equivalence function given by the exclusive-NOR operation (XNOR),
a function can be formed whose output is logic-1 if and only if both $f$ and $f_c$ are at
logic-1, $\overline{f \oplus f_c}$.  Furthermore, if the output probability of this function
is computed we have the percentage of outputs where both $f$ and $f_c$ simultaneously
output the same value, $p(\overline{f \oplus f_c})$.  The corresponding Walsh
spectral coefficient can then be obtained by scaling the output probability value by
$2^n$, where $n$ is the number of variables in $f$.  This is given in Equation
\ref{eq:walsh_1}.

\begin{equation}
W_f(f_c) = 2^n [ 1-p(f \oplus f_c)]    \label{eq:walsh_1}
\end{equation}

The Walsh coefficients in Equation \ref{eq:walsh_1} depend upon the equivalence relation of
the function being transformed and the constituent function.  However, the spectral 
coefficient can be computed using Boolean operators other than XOR.  This is accomplished by
exploiting the probability relationships given in Equations \ref{eq:xor_andor}
and \ref{eq:and_or}.

\begin{equation}
p(f+f_c)=p(f \cdot f_c)+p(f \oplus f_c)   \label{eq:xor_andor}
\end{equation}

\begin{equation}
p(f+f_c)=p(f)+p(f_c)-p(f \cdot f_c)   \label{eq:and_or}
\end{equation}

Substituting the relationships in Equations \ref{eq:xor_andor} and 
\ref{eq:and_or} into Equation 
\ref{eq:walsh_1} yields the following results.

\begin{equation}
W_f(f_c) = 2^n [1+2 (p(f \cdot f_c)-p(f+f_c))]  \label{eq:walsh_2}
\end{equation}

\begin{equation}
W_f(f_c) = 2^n[1+4p(f \cdot f_c)-2p(f)-2p(f_c)] \label{eq:walsh_3}
\end{equation}

\begin{equation}
W_f(f_c) = 2^n[1-4p(f + f_c)+2p(f)+2p(f_c)] \label{eq:walsh_4}
\end{equation}

The relationships in Equations \ref{eq:walsh_3} and \ref{eq:walsh_4} may be simplified 
since $p(f_c)=\frac{1}{2}$ is easily shown to be true for all $f_c$ except
the zeroth ordered coefficient, $W_f(0)$, where $p(0)=0$.  In addition, the 
higher ordered Walsh coefficients can be related to the the zeroth ordered coefficient
through the expressions given in Equations \ref{eq:walsh_5},  \ref{eq:walsh_6}, and
\ref{eq:walsh_7}.

\begin{equation}
W_f(0)=2^n(1-2p(f))     \label{eq:walsh_5}
\end{equation}

\begin{equation}
W_f(f_c) = 2^n ( 4p(f \cdot f_c) - 1) + W_f(0)    \label{eq:walsh_6}
\end{equation}

\begin{equation}
W_f(f_c)=2^n(3-4p(f + f_c))-W_f(0)    \label{eq:walsh_7}
\end{equation}

These relations may be rearranged to compute the output probabilities directly from
the Walsh coefficients if so desired.

\subsection{Reed-Muller Spectrum}
The Reed-Muller family of spectra consist of $2^n$ distinct transformations.
These are generally classified according to a {\em polarity number}.  The polarity
number is used to indicate if a literal is present in complemented or non-complemented
form in the generalized Reed-Muller Boolean algebraic expression. In terms of the
Reed-Muller spectra, the polarity number can be considered to uniquely define the
transformation matrix.  The generalized Reed-Muller spectra have been studied
and used extensively in the past.  References \cite{DDT78} and \cite{DG86} provide
excellent background material.

In relating output probabilities to the Reed-Muller spectra considerations must
be made due to the fact that the RM spectrum is computed over Galois field 2 and
the output probabilities are real quantities in the interval [0,1].  An isomorphic
relation is used to address this problem \cite{TN95b}.  Like the
Walsh spectrum, each of the rows of the RM transformation matrix may be viewed as
constituent function output vectors.  The constituent functions turn out to be
all possible products of literals (with complementation or lack thereof) determined
by the polarity number.  Therefore, any arbitrary RM spectral coefficient may
be computed as given in Equation \ref{eq:rm1}.

\begin{equation}
R_f(f_c) = [2^np(f \cdot f_c)] (\bmod{2})   \label{eq:rm1}
\end{equation}

Using the probability relationships given in Equations \ref{eq:xor_andor} and 
\ref{eq:and_or}, alternative relationships can be derived as given in the
following.

\begin{equation}
R_f(f_c) = [2^n (p(f)-p(f_c)-p(f+g))] (\bmod{2}) \label{eq:rm2}
\end{equation}

\begin{equation}
R_f(f_c) = [2^n (p(f \oplus f_c)-p(f+g))] (\bmod{2}) \label{eq:rm3}
\end{equation}

\subsection{Haar Spectrum}
The Haar spectrum \cite{AR75} \cite{SH81} \cite{MK76} 
has not seen as much use as the Walsh family or RM transforms
described above.  This is likely due to the fact that the transformation matrix
is not symmetric. However, the Haar spectrum and
in particular the modified Haar spectrum can provide very interesting information
concerning the correlation of a function with various Shannon decompositions \cite{HMM85}.  The
Shannon decompositions have seen quite a lot of scrutiny with the recent popularity
of Decision Diagram (DD) representations of Boolean functions since the DD form can
often be defined in terms of Shannon decompositions.

The row vectors of the Haar matrix can in fact be described in terms of the AND
operation of the function to be transformed and various cubes formed from the 
literals.  This is in contrast to the Walsh and RM transformation matrices where the
respective $f_c$ functions were defined with total independence of the function to
be transformed.  The reason for this difference in row vector definition
is because the Haar transform
does not give totally global information, rather it gives information regarding
the correlation of a function with its various cofactors.  However, like the previous
approaches, each Haar coefficient can be computed as an algebraic relation of various
probability values and hence the Haar spectrum is also directly linked with output
probability calculations.

Like the Walsh family of transforms, the output vector of the function to be transformed
generally contains integers with -1 representing logic-1 and +1 representing logic-0.
With this viewpoint, we can define the number of matches between a particular transformation
matrix row vector as the number of times the row vector and function vector components are simultaneously
equal to -1 or +1.  As an example of the modified Haar transform, consider the following 
transformation matrix for a 3-variable function is shown in Figure \ref{haar_mat}.  
Each of the constituent Boolean functions are
given to the left of their respective row vectors.  Note that the matrix contains a 0 value
in addition to the 1 and -1 quantities which represent logic levels.  Since some of the
rows represent constituent functions that are cofactors, the output space is less than $2^3$
and the presence of a 0 value acts as a place holder.

\begin{figure}
\begin{flushleft}
\[
\begin{array}{rl}
\begin{array}{r}
f \\
x_1 \\
x_2 \cdot f_{\overline{x}_1} \\
x_2 \cdot f_{x_1} \\
x_3 \cdot f_{\overline{x}_1 \overline{x}_2} \\
x_3 \cdot f_{\overline{x}_1 x_2} \\
x_3 \cdot f_{ x_1 \overline{x}_2} \\
x_3 \cdot f_{x_1 x_2} \\
\end{array} &
\! \! \! \! \! \! \! \! \! \left[ \! \!
\begin{array}{rrrrrrrr}
1 & \! 1 & \! 1 & \! 1 & \! 1 & \! 1 & \! 1 & \! 1 \\
1 & \! 1 & \! 1 & \! 1 & \! -1 & \! -1 & \! -1 & \! -1 \\
1 & \! 1 & \! -1 & \! -1 & \! 0 & \! 0 & \! 0 & \! 0 \\
0 & \! 0 & \! 0 & \! 0 & \! 1 & \! 1 & \! -1 & \! -1 \\
1 & \! -1 & \! 0 & \! 0 & \! 0 & \! 0 & \! 0 & \! 0 \\
0 & \! 0 & \! 1 & \! -1 & \! 0 & \! 0 & \! 0 & \! 0 \\
0 & \! 0 & \! 0 & \! 0 & \! 1 & \! -1 & \! 0 & \! 0 \\
0 & \! 0 & \! 0 & \! 0 & \! 0 & \! 0 & \! 1 & \! -1 \\
\end{array} \! \! \right]
\end{array} \]
\end{flushleft}
\caption{Example of Modified Haar Transformation Matrix for $n=3$} \label{haar_mat}
\end{figure}

By applying the same principles as those used to determine the algebraic relationships
between output probabilities and the Walsh family and RM transforms, we can formulate the
Haar transform relationships.  The presence of cofactors in the Haar constituent functions
can be handled by using the relationship in Equation \ref{eq:baye} to represent these quantities
as output probabilities of the AND of the function to be transformed with its respective
dependent literals.  Note also that the maximum absolute value of a Haar spectral coefficient
varies depending on the order of the coefficient.  This is due to the reduction in 
the size of the range of the constituent functions containing cofactors.  The following table
contains symbols for each of the Haar coefficients, $H_i$, values that indicate the size of the
constituent function range, $i$ and $n$, and probability expressions that evaluate whether the
function to be transformed and the constituent function simultaneously evaluate to
logic-0, $p_{m0}$, or, evaluate to logic-1, $p_{m1}$.

\renewcommand{\arraystretch}{1.3}
\begin{table} 
\caption{Relationship of the Haar Spectrum and Output Probabilities} \label{haar_tab}
\begin{center}
\begin{tabular}{|c|c|c|c|c|}  \hline
{\bf SYMBOL} & {\bf $i$} & {\bf $n$ } & {\bf $p_{m0}$ } & {\bf $p_{m1}$ } \\ \hline \hline
$H_0$ & 0 & 3 & $p(f \cdot 0)$ & $p(\overline{f} \cdot \overline{0})$ \\ \hline
$H_1$ & 0 & 3 & $p(f \cdot x_1)$ & $p(\overline{f} \cdot \overline{x}_1)$ \\ \hline
$H_2$ & 1 & 2 & $p(f \cdot \overline{x_1} \cdot x_2) /p(\overline{x}_1)$ & $p(\overline{f \cdot \overline{x}_1} \cdot \overline{x}_2) /p(\overline{x}_1)$ \\ \hline
$H_3$ & 1 & 2 & $p(f \cdot x_1 \cdot x_2) /p(x_1)$ & $p(\overline{f \cdot x_1} \cdot \overline{x}_2) /p(x_1)$ \\ \hline
$H_4$ & 2 & 1 & $p(f \cdot \overline{x}_1 \cdot \overline{x}_2 \cdot x_3) /p( \overline{x}_1 \cdot \overline{x}_2)$ & $p(\overline{f \cdot \overline{x}_1 \cdot \overline{x}_2} \cdot \overline{x}_3) /p(\overline{x}_1 \cdot \overline{x}_2)$ \\ \hline
$H_5$ & 2 & 1 & $p(f \cdot \overline{x}_1 \cdot {x}_2 \cdot x_3) /p(\overline{x}_1 \cdot x_2)$ & $p(\overline{f \cdot \overline{x}_1 \cdot x_2} \cdot \overline{x}_3) /p(\overline{x}_1 \cdot x_2)$ \\ \hline
$H_6$ & 2 & 1 & $p(f \cdot x_1 \cdot \overline{x}_2 \cdot x_3) /p(x_1 \cdot \overline{x}_2)$ & $p(\overline{f \cdot x_1 \cdot \overline{x}_2} \cdot \overline{x}_3) /p(x_1 \cdot \overline{x}_2)$ \\ \hline
$H_7$ & 2 & 1 & $p(f \cdot x_1 \cdot x_2 \cdot x_3) /p(x_1 \cdot x_2)$ & $p(\overline{f \cdot x_1 \cdot x_2} \cdot \overline{x}_3) /p(x_1 \cdot x_2)$ \\ \hline \hline
\end{tabular}
\end{center}
\end{table}

Using the notation introduced in Table \ref{haar_tab}, we can write an algebraic equation
to compute the value of a Haar spectral coefficient in terms of the output probabilities
used to compute $p_{m0}$ and $p_{m1}$.

\begin{equation}
H = 2^{n-i} [ 2 ( p_{m0} + p_{m1} ) - 1 ] \label{eq:haar_eqn}
\end{equation}

