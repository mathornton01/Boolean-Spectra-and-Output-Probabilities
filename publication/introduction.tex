\section{Introduction}
Various types of spectra of a Boolean function provide a unique 
means of definition, and,
each individual spectral coefficient provides information
regarding the behavior of the function over all possible inputs.
On the other extreme, the function may also be completely
described by a vector of Boolean values corresponding to the
output of the function for each possible variable assignment.
This paper will 
show how circuit output probabilities can be used as fundamental
quantities that have simple algebraic relations with both various
spectral coefficients and Boolean outputs of a digital logic function.

Circuit output probabilities \cite{PM75} were first proposed
for the analysis of the usefulness of random testing.  Recently,
these values 
have found renewed interest due to their close relationship with
`signal switching probabilities', or `switching activity factors'
which are used extensively in the development and analysis of
low power circuitry \cite{GDKW92}  \cite{MP96} \cite{TPD93}.  

This relationship is intriguing since it allows the global behavior
provided by the spectral coefficients and the local behavior 
specified by the truth table output vector to be related through a common
set of values. 
These relations can be used to develop 
an efficient means for computing the
Walsh family of spectra \cite{TN94e} and generalized Reed-Muller
spectra \cite{TN95a}.  In previous work, the emphasis was
to find efficient ways to compute individual spectral coefficients
by quickly computing a specific set of circuit output probabilities.
The results presented here are a result of the generalization of these ideas
and thus the main results focus on
an examination of the inherent relationship between
output probabilities and other forms of representation.

In addition to using the output probability values to represent a 
specific Boolean function, it is also shown that properties of the
functions such as smoothing, cofactors, consensus, and the Boolean
difference may also be represented by their respective output
probabilities.  In particular, by using the properties of conditional
probabilities, we show that only 4 values are required for 
the computation of the corresponding output probabilities of these 
functions.

This paper is organized as follows.  First, a brief explanation of methods 
for computing output probabilities will be described.
Next, 
the development of the algebraic relationship between 
Boolean function output probabilities and corresponding values in the
Boolean and spectral domains is presented.  This development will include several different
types of spectra.  
Finally the paper will include a concluding
section that contains a summary of the work presented and future
planned efforts.
